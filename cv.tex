%!TEX TS-program = xelatex
%!TEX encoding = UTF-8 Unicode
% Awesome CV LaTeX Template for CV/Resume
%
% This template has been downloaded from:
% https://github.com/posquit0/Awesome-CV
%
% Author:
% Claud D. Park <posquit0.bj@gmail.com>
% http://www.posquit0.com
%
% Modifications by:
% Trevor Rife (trife@ksu.edu)
%
% Template license:
% CC BY-SA 4.0 (https://creativecommons.org/licenses/by-sa/4.0/)
%


%-------------------------------------------------------------------------------
% CONFIGURATIONS
%-------------------------------------------------------------------------------
% A4 paper size by default, use 'letterpaper' for US letter
\documentclass[11pt, letterpaper, academicons]{awesome-cv}

% Configure page margins with geometry
\geometry{left=1.4cm, top=1.4cm, right=1.4cm, bottom=1.8cm, footskip=.5cm}

% Specify the location of the included fonts
\fontdir[fonts/]

% Color for highlights
% Awesome Colors: awesome-emerald, awesome-skyblue, awesome-red, awesome-pink, awesome-orange
%                 awesome-nephritis, awesome-concrete, awesome-darknight
%\colorlet{awesome}{awesome-red}
% Uncomment if you would like to specify your own color
\definecolor{awesome}{HTML}{CC0000}

% kansas state purple: 512888
% clemson orange: F56600
% north carolina state red: CC0000


% Colors for text
% Uncomment if you would like to specify your own color
% \definecolor{darktext}{HTML}{414141}
% \definecolor{text}{HTML}{333333}
% \definecolor{graytext}{HTML}{5D5D5D}
% \definecolor{lighttext}{HTML}{999999}

% Set false if you don't want to highlight section with awesome color
\setbool{acvSectionColorHighlight}{true}

% If you would like to change the social information separator from a pipe (|) to something else
\renewcommand{\acvHeaderSocialSep}{\quad\textbar\quad}

\makeatletter
\patchcmd{\@sectioncolor}{\color}{\mdseries\color}{}{}
\makeatother

% For presentations and publications
\usepackage{bibentry}

%% sample.bib contains your publications
%\addbibresource{ref.bib}


%-------------------------------------------------------------------------------
%	PERSONAL INFORMATION
%	Comment any of the lines below if they are not required
%-------------------------------------------------------------------------------
% Available options: circle|rectangle,edge/noedge,left/right
% \photo{./examples/profile.png}
\name{Trevor W.}{Rife}
\position{Assistant Professor}
\address{Pee Dee Research \& Education Center\\2200 Pocket Road\\Clemson University\\Florence, SC 29506}

%\mobile{(307) 575-4033}
\email{twrife@clemson.edu}
\homepage{www.rifelab.org}
\github{trife}
\linkedin{trevor-rife}
% \gitlab{gitlab-id}
% \stackoverflow{SO-id}{SO-name}
% \twitter{@TrevorRife}
% \skype{skype-id}
% \reddit{reddit-id}
% \medium{madium-id}
\googlescholar{m4sw8RQAAAAJ}{T.W. Rife}
%% \firstname and \lastname will be used
% \googlescholar{googlescholar-id}{}
% \extrainfo{}
\orcid{0000-0002-5974-6523}

%-------------------------------------------------------------------------------
%	BIBLIOGRAPHIES
%-------------------------------------------------------------------------------
\addbibresource{bib/bib_articles.bib}
\addbibresource{bib/bib_software.bib}
\addbibresource{bib/bib_preprints.bib}
\addbibresource{bib/bib_posters.bib}
\addbibresource{bib/bib_talks.bib}

%-------------------------------------------------------------------------------
\begin{document}
% Print the header with above personal information
% Give optional argument to change alignment(C: center, L: left, R: right)
\makecvheader
% Print the footer with 3 arguments(<left>, <center>, <right>)
% Leave any of these blank if they are not needed
\makecvfooter
  {\today}
  {TW Rife~~~·~~~Curriculum Vitae}
  {\thepage}

%-------------------------------------------------------------------------------
%	CV/RESUME CONTENT
%	Each section is imported separately, open each file in turn to modify content
%-------------------------------------------------------------------------------
%-------------------------------------------------------------------------------
%	SECTION TITLE
%-------------------------------------------------------------------------------
\cvsection{Education}

%-------------------------------------------------------------------------------
%	CONTENT
%-------------------------------------------------------------------------------
\begin{cventries}

%---------------------------------------------------------
  \cventry
    {Ph.D. in Genetics} % Degree
    {Kansas State University} % Institution
    {Manhattan, KS} % Location
    {Jul. 2011 - Dec. 2016} % Date(s)
    {
      \begin{cvitems} % Description(s) bullet points
        \item {\textbf{Thesis:} Utilizing a historical wheat collection to develop new tools for modern plant breeding}
        \item {\textbf{Advisor:} Jesse A. Poland}
      \end{cvitems}
    }

%---------------------------------------------------------
  \cventry
    {B.S. in Molecular Biology, \textit{Cum Laude}} % Degree
    {University of Wyoming} % Institution
    {Laramie, WY} % Location
    {Sept. 2008 - May 2011} % Date(s)
    {}
%---------------------------------------------------------
    
\end{cventries}
%-------------------------------------------------------------------------------
%	SECTION TITLE
%-------------------------------------------------------------------------------
\cvsection{Appointments}

\begin{cvskills}

%---------------------------------------------------------
  \cvskill
    {2022 - Present} % Category
    {Assistant Professor, Clemson University} % Skills

%---------------------------------------------------------
  \cvskill
    {2021 - 2022} % Category
    {Research Assistant Professor, Kansas State University} % Skills

%---------------------------------------------------------
  \cvskill
    {2021 - 2022} % Category
    {Director, Feed the Future Innovation Lab for Applied Wheat Genomics} % Skills

%---------------------------------------------------------
\end{cvskills}
%-------------------------------------------------------------------------------
%	SECTION TITLE
%-------------------------------------------------------------------------------
\cvsection{Expertise \& Interests}


%-------------------------------------------------------------------------------
%	CONTENT
%-------------------------------------------------------------------------------
\begin{cvskills}

%---------------------------------------------------------
  \cvskill
    {Plant Breeding} % Category
    {Experimental field design and evaluation, Genomic prediction, International development} % Skills

%---------------------------------------------------------
  \cvskill
    {Phenomics} % Category
    {PhenoApps, Field Book, Image analysis, Data collection and management} % Skills

%---------------------------------------------------------
  \cvskill
    {Genomics} % Category
    {Genotyping-by-sequencing, Marker development, Bioinformatics and workflow optimization} % Skills

%---------------------------------------------------------
  \cvskill
    {Programming} % Category
    {Java, R, Linux, Android, SQL} % Skills

%---------------------------------------------------------
\end{cvskills}

%-------------------------------------------------------------------------------
%	SECTION TITLE
%-------------------------------------------------------------------------------
\cvsection{Research Experience}

%-------------------------------------------------------------------------------
%	CONTENT
%-------------------------------------------------------------------------------
\begin{cventries}

  \cventry
    {Postdoctoral Fellow} % Job title
    {Kansas State University} % Organization
    {Manhattan, KS} % Location
    {Dec. 2016 - Mar. 2021} % Date(s)
    {
      \begin{cvitems} % Description(s) of tasks/responsibilities
        \item {Designed and developed mobile applications for phenotypic data collection}
        \item {Led training workshops in Nigeria, Uganda, Tanzania, Bangladesh, Nepal, and India for PhenoApps that trained 150+ scientists}
        \item {Developed project website and training modules: \href{www.PhenoApps.org}{\textit{PhenoApps.org}}}
        \item {Managed a team of three developers}
      \end{cvitems}
    }

%---------------------------------------------------------
  \cventry
    {Graduate Research Assistant} % Job title
    {Kansas State University} % Organization
    {Manhattan, KS} % Location
    {Jul. 2011 - Dec. 2016} % Date(s)
    {
      \begin{cvitems} % Description(s) of tasks/responsibilities
        \item {Utilized historical data (22 years; 939 entries; 100k plots) to validate genomic selection as a  tool for regional plant breeding}
        \item {Organized and managed multi-location field experiments to estimate the current rate of genetic gain}
        \item {Developed a marker platform for targeted allele genotyping}
        \item {Designed and developed Field Book mobile app for plant scientists}
      \end{cvitems}
    }

%---------------------------------------------------------
  \cventry
    {Assistant Plant Breeder} % Job title
    {High Plains Crop Development} % Organization
    {Torrington, WY} % Location
    {May 2006 - Jul. 2011} % Date(s)
    {
      \begin{cvitems} % Description(s) of tasks/responsibilities
        \item {Planted, harvested, crossed, and selected canola varieties}
      \end{cvitems}
    }

%---------------------------------------------------------
  \cventry
    {Undergraduate Research Assistant} % Job title
    {University of Wyoming (Dr. Dan Wall Lab)} % Organization
    {Laramie, WY} % Location
    {Jan. 2010 - May 2011} % Date(s)
    {
      \begin{cvitems} % Description(s) of tasks/responsibilities
        \item {Characterized NTRC-like activator mutants in \textit{Myxococcus xanthus}}
      \end{cvitems}
    }

%---------------------------------------------------------
  \cventry
    {Undergraduate Research Assistant} % Job title
    {University of Wyoming (Dr. Peter Thorsness Lab)} % Organization
    {Laramie, WY} % Location
    {Jan. 2009 - May 2009} % Date(s)
    {
      \begin{cvitems} % Description(s) of tasks/responsibilities
        \item {Generated and characterized BAX mutants in \textit{Saccharomyces cerevisiae}}
      \end{cvitems}
    }

%---------------------------------------------------------
\end{cventries}

%-------------------------------------------------------------------------------
%	SECTION TITLE
%-------------------------------------------------------------------------------
\cvsection{Grants}
\begin{cvgrants}

%---------------------------------------------------------
  \cvgrant
    {co-PI, \$806,112 (\$25,000,000)} 
    {Feed the Future Innovation Lab for Crop Improvement}
    {USAID} % Location
    {2019 - 2024} % Date(s)
    {
    }

%---------------------------------------------------------
  \cvgrant
    {PI, \$249,971 (\$500,000)} % Degree
    {FACT: Geospatial Plant Breeding Tools and Technologies} % Institution
    {USDA NIFA} % Location
    {2019 - 2022} % Date(s)
    {
    }

%---------------------------------------------------------
  \cvgrant
    {co-PI, \$714,804 (\$1,632,824)} % Degree
    {BREAD PHENO: High-Throughput Phenotyping with Smart Phones. \#phenoApps} % Institution
    {NSF BREAD} % Location
    {2016 - 2020} % Date(s)
    {
    }

%---------------------------------------------------------
  \cvgrant
    {co-PI, \$5,500} % Degree
    {Developing Smartphone Applications for Estimating the Grain Yield of Winter Wheat} % Institution
    {KS Wheat Alliance} % Location
    {2013 - 2014} % Date(s)
    {
    }

%---------------------------------------------------------
\end{cvgrants}
%-------------------------------------------------------------------------------
%	PUBLICATIONS
%-------------------------------------------------------------------------------
\cvsection{Publications}

%-------------------------------------------------------------------------------
%	PREPRINTS
%-------------------------------------------------------------------------------
\cvsubsection{Preprints \& Upcoming}

\begin{refsection}
	\nocite{rife_onekk}

	\newrefcontext[
    sorting=none
    ]

	\printbibliography[
	heading=none
	]
\end{refsection}

%-------------------------------------------------------------------------------
%	JOURNAL ARTICLES
%-------------------------------------------------------------------------------
\cvsubsection{Journal Articles}

\begin{refsection}
	\nocite{Poland2012}
	\nocite{Rife2014}
	\nocite{Rife2015}
	\nocite{Tack2016}
	\nocite{Neilsen2016}
	\nocite{Neilsen2017}
	\nocite{Courtney2017}
	\nocite{Shao2018}
	\nocite{Rife2018}
	\nocite{Rife2019}
	\nocite{Selby2019}
	\nocite{Rife2021}
	\nocite{Gao2021}
	\nocite{Rife2022}
	\nocite{Morales2022}
	\nocite{Margapuri2022}

	\printbibliography[
	heading=none
	]
\end{refsection}

%-------------------------------------------------------------------------------
%	Software
%-------------------------------------------------------------------------------
\cvsubsection{Software}

\begin{refsection}
	\nocite{field_book}
	\nocite{coordinate}
	\nocite{inventory}
	\nocite{intercross}
	\nocite{verify}
	\nocite{prospector}
	\nocite{onekk}

    \newrefcontext[
    sorting=none
    ]

	\printbibliography[
	heading=none
	]
\end{refsection}

%-------------------------------------------------------------------------------
%	SECTION TITLE
%-------------------------------------------------------------------------------
\cvsection{Presentations}

%-------------------------------------------------------------------------------
%	CONFERENCE TALKS
%-------------------------------------------------------------------------------
\cvsubsection{Invited Talks}
\begin{refsection}
	\nocite{rife_pag_2023_b}
	\nocite{rife_pag_2023_a}
	\nocite{rife_cornell_2021}
	\nocite{rife_pag_2019}
	\nocite{rife_ashs_2018}
	\nocite{rife_bgri_2018}
	\nocite{rife_ashs_2017}
	\nocite{rife_pag_2016}
	\nocite{rife_ksu_2014}
	\nocite{rife_pag_2014}
	
	\printbibliography[
	heading=none
	]
\end{refsection}

%-------------------------------------------------------------------------------
%	CONFERENCE POSTERS
%-------------------------------------------------------------------------------
\cvsubsection{Conference Posters}
\begin{refsection}
	\nocite{rife_pag_2013_poster}
	\nocite{rife_pag_2014_poster}
	%\nocite{rife_borlaug_2014_poster1}
	\nocite{rife_borlaug_2014_poster2}
	\nocite{rife_iwc_2015_poster1}
	\nocite{rife_iwc_2015_poster2}
	\nocite{rife_pag_2016_poster1}
	\nocite{rife_pag_2016_poster2}
	\nocite{rife_napb_2016_poster}
	\nocite{rife_napb_2018_poster}
	\nocite{rife_pag_2019_poster}

	\printbibliography[
	heading=none
	]
\end{refsection}
\break
%-------------------------------------------------------------------------------
%	SECTION TITLE
%-------------------------------------------------------------------------------
\cvsection{Workshops}

%-------------------------------------------------------------------------------
%	CONTENT
%-------------------------------------------------------------------------------
\begin{cvworkshops}

%---------------------------------------------------------
  \cvworkshop
    {ILCI JupyterHub Interoperability} % Committee
    {Saly, Senegal} % Location
    {Oct. 2022} % Date(s)
%---------------------------------------------------------
  \cvworkshop
    {Utilizing PhenoApps for plant breeding} % Committee
    {Bhairahawa, Nepal} % Location
    {Mar. 2020} % Date(s)
%---------------------------------------------------------
  \cvworkshop
    {Utilizing PhenoApps for plant breeding} % Committee
    {Kathmandu, Nepal} % Location
    {Mar. 2020} % Date(s)
%---------------------------------------------------------
  \cvworkshop
    {Utilizing PhenoApps for cassava breeding} % Committee
    {Kampala, Uganda} % Location
    {Feb. 2019} % Date(s)
%---------------------------------------------------------
  \cvworkshop
    {PhenoApps: Open-source apps for plant breeding and genetics} % Committee
    {Dinajpur, Bangladesh} % Location
    {Feb. 2019} % Date(s)
%---------------------------------------------------------
  \cvworkshop
    {PhenoApps Hackathon} % Committee
    {Ithaca, NY} % Location
    {Jul. 2018} % Date(s)
%---------------------------------------------------------
  \cvworkshop
    {Utilizing PhenoApps for cassava breeding} % Committee
    {Dar es Salaam, Tanzania} % Location
    {Feb. 2018} % Date(s)
%---------------------------------------------------------
  \cvworkshop
    {PhenoApps for genotypic sample collection and tracking} % Committee
    {Hyderabad, India} % Location
    {Dec. 2017} % Date(s)
%---------------------------------------------------------
  \cvworkshop
    {PhenoApps: Open-source apps for plant breeding and genetics} % Committee
    {Kampala, Uganda} % Location
    {Nov. 2017} % Date(s)
%---------------------------------------------------------
  \cvworkshop
    {PhenoApps: Open-source apps for plant breeding and genetics} % Committee
    {Ludhiana, India} % Location
    {Oct. 2017} % Date(s)
%---------------------------------------------------------
  \cvworkshop
    {Field Book: An open-source Android app for collecting phenotypic data} % Committee
    {Manhattan, KS} % Location
    {Mar. 2016} % Date(s)
%---------------------------------------------------------
  \cvworkshop
    {Using Field Book for data collection in agriculture} % Committee
    {Bishoftu, Ethiopia} % Location
    {Mar. 2015} % Date(s)
%---------------------------------------------------------
  \cvworkshop
    {Using Field Book for data collection in agriculture} % Committee
    {Rongo, Kenya} % Location
    {Mar. 2015} % Date(s)
%---------------------------------------------------------
  \cvworkshop
    {Using Field Book for data collection in agriculture} % Committee
    {Kakamega, Kenya} % Location
    {Mar. 2015} % Date(s)
%---------------------------------------------------------
  \cvworkshop
    {Using Android Field Book for data collection} % Committee
    {Addis Ababa, Ethiopia} % Location
    {May 2013} % Date(s)
%---------------------------------------------------------
  \cvworkshop
    {One Handheld Per Breeder Introduction} % Committee
    {Nairobi, Kenya} % Location
    {Nov. 2012} % Date(s)
%---------------------------------------------------------
\end{cvworkshops}

%-------------------------------------------------------------------------------
%	SECTION TITLE
%-------------------------------------------------------------------------------
\cvsection{Service \& Membership}

%-------------------------------------------------------------------------------
%	CONTENT
%-------------------------------------------------------------------------------
\begin{cventries}

%---------------------------------------------------------
  \cventry
    {Member} % Affiliation/role
    {Crop Science Society of America} % Organization/group
    {} % Location
    {2011 - Present} % Date(s)
    {}

%---------------------------------------------------------
  \cventry
    {Member} % Affiliation/role
    {National Association of Plant Breeders} % Organization/group
    {} % Location
    {2015 - Present} % Date(s)
    {}

%---------------------------------------------------------
  \cventry
    {Member} % Affiliation/role
    {American Academy of Arts and Sciences} % Organization/group
    {} % Location
    {2014 - Present} % Date(s)
    {}

%---------------------------------------------------------
  \cventry
    {Co-Founder (2012), President (2013-2014), Treasurer (2014-2016)} % Affiliation/role
    {KSU Plant Breeding \& Genetics Club} % Organization/group
    {} % Location
    {2012 - 2016} % Date(s)
    {
      \begin{cvitems} % Description(s) of experience/contributions/knowledge
        \item {2015: Symposium Organizing Committee – \textit{Heterosis: Foundation of Food, Fuel, and Fiber}}
        \item {2013: Symposium Organizing Committee – \textit{Next Generation Plant Breeding}}
      \end{cvitems}
    }

%---------------------------------------------------------
  \cventry
    {Treasurer} % Affiliation/role
    {KSU Plant Pathology Graduate Student Club } % Organization/group
    {} % Location
    {2012 - 2013} % Date(s)
    {}
%---------------------------------------------------------
\end{cventries}

%-------------------------------------------------------------------------------
%	SECTION TITLE
%-------------------------------------------------------------------------------
\cvsection{Editorial}

%-------------------------------------------------------------------------------
%	CONTENT
%-------------------------------------------------------------------------------
\begin{cventries}

%---------------------------------------------------------
  \cventry
    {The Plant Phenome Journal} % Organization/group
    {Technical Editor} % Affiliation/role
    {} % Location
    {2024 - Present} % Date(s)
    {}

%---------------------------------------------------------
  \cventry
    {The Plant Phenome Journal} % Organization/group
    {Associate Editor} % Affiliation/role
    {} % Location
    {2023 - Present} % Date(s)
    {}

  %---------------------------------------------------------
    \cventry
      {USDA NIFA AFRI} % Affiliation/role
      {Grant Reviewer} % Organization/group
      {} % Location
      {2021} % Date(s)
      {}
    
%---------------------------------------------------------
  \cventry
    {The Plant Genome, Crop Science, Theoretical and Applied Genetics, Plant Disease, G3, Plant Physiology, The Plant Phenome Journal, Agricultural Systems} % Affiliation/role
    {Ad Hoc Reviewer} % Organization/group
    {} % Location
    {} % Date(s)
    {}

%---------------------------------------------------------
\end{cventries}

%-------------------------------------------------------------------------------
%	SECTION TITLE
%-------------------------------------------------------------------------------
\cvsection{Volunteer}

%-------------------------------------------------------------------------------
%	CONTENT
%-------------------------------------------------------------------------------
\begin{cventries}

%---------------------------------------------------------
  \cventry
    {Board Member} 
    {Social Services Advisory Board}
    {Manhattan, KS} % Location
    {2017 - 2022} % Date(s)
    {
    }

%---------------------------------------------------------
  \cventry
    {Foster Home} % Degree
    {Riley County Humane Society} % Institution
    {Manhattan, KS} % Location
    {2018 - 2022} % Date(s)
    {
    }

%---------------------------------------------------------
\end{cventries}
%-------------------------------------------------------------------------------
%	SECTION TITLE
%-------------------------------------------------------------------------------
\cvsection{Scholarships \& Awards}

\begin{cvawards}
%---------------------------------------------------------
  \cvaward
    {Crop Science Society of America Poster Contest, Third Place} % Award
    {Kansas State University} % Location
    {2017} % Date(s)
%---------------------------------------------------------
  \cvaward
    {Don C. Warren Genetics Scholarship} % Award
    {Kansas State University} % Location
    {2016} % Date(s)
%---------------------------------------------------------
  \cvaward
    {KSU Graduate Student Council Travel Grant} % Award
    {Kansas State University} % Location
    {2016} % Date(s)
%---------------------------------------------------------
  \cvaward
    {Don C. Warren Genetics Scholarship} % Award
    {Kansas State University} % Location
    {2015} % Date(s)
%---------------------------------------------------------
  \cvaward
    {ASA-CSSA-SSSA Future Leaders in Science Award} % Award
    {Kansas State University} % Location
    {2015} % Date(s)
%---------------------------------------------------------
  \cvaward
    {Triticeae Coordinated Agriculture Project Travel Scholarship} % Award
    {Kansas State University} % Location
    {2014} % Date(s)
%---------------------------------------------------------
  \cvaward
    {KSU Graduate Student Council Travel Grant} % Award
    {Kansas State University} % Location
    {2014} % Date(s)
%---------------------------------------------------------
  \cvaward
    {8th International Purdue Symposium on Statistics Travel Scholarship} % Award
    {Kansas State University} % Location
    {2012} % Date(s)
%---------------------------------------------------------
  \cvaward
    {KSU Graduate Student Council Travel Grant} % Award
    {Kansas State University} % Location
    {2012} % Date(s)
%---------------------------------------------------------
  \cvaward
    {Trustees' Pride Scholars Award} % Award
    {University of Wyoming} % Location
    {2008} % Date(s)
%---------------------------------------------------------
  \cvaward
    {Honors Hathaway Scholarship} % Award
    {University of Wyoming} % Location
    {2008} % Date(s)
%---------------------------------------------------------
  \cvaward
    {Likins Memorial Scholarship} % Award
    {University of Wyoming} % Location
    {2008} % Date(s)

\end{cvawards}



\break
%-------------------------------------------------------------------------------
%	SECTION TITLE
%-------------------------------------------------------------------------------
\cvsection{References}

%-------------------------------------------------------------------------------
%	CONTENT
%-------------------------------------------------------------------------------
\begin{cventries}

%---------------------------------------------------------
  \cventry
    {King Abdullah University of Science and Technology} % Degree
    {Jesse Poland} % Institution
    {} % Location
    {} % Date(s)
    {
      \begin{cvitems} % Description(s) bullet points
        \item {Professor}
        \item {jpoland@ksu.edu}
        \item {785-532-2709}
      \end{cvitems}
    }

%---------------------------------------------------------
  \cventry
    {Cornell University} % Degree
    {Michael Gore} % Institution
    {} % Location
    {} % Date(s)
    {
      \begin{cvitems} % Description(s) bullet points
        \item {Professor and Chair}
        \item {mag87@cornell.edu}
        \item {607-255-5492}
      \end{cvitems}
    }

%---------------------------------------------------------
  \cventry
    {Kansas State University} % Degree
    {Allan Fritz} % Institution
    {} % Location
    {} % Date(s)
    {
      \begin{cvitems} % Description(s) bullet points
        \item {Professor}
        \item {akf@ksu.edu}
        \item {785-532-7245}
      \end{cvitems}
    }

%---------------------------------------------------------
  \cventry
    {Boyce Thompson Institute} % Degree
    {Lukas Mueller} % Institution
    {} % Location
    {} % Date(s)
    {
      \begin{cvitems} % Description(s) bullet points
        \item {Professor}
        \item {lam87@cornell.edu}
        \item {607-255-6557}
      \end{cvitems}
    }

%---------------------------------------------------------
  \cventry
    {Kansas State University} % Degree
    {Mitchell Neilsen} % Institution
    {} % Location
    {} % Date(s)
    {
      \begin{cvitems} % Description(s) bullet points
        \item {Professor}
        \item {neilsen@k-state.edu}
        \item {785-532-7918}
      \end{cvitems}
    }

%---------------------------------------------------------
  \cventry
    {USDA-ARS} % Degree
    {Jean-Luc Jannink} % Institution
    {} % Location
    {} % Date(s)
    {
      \begin{cvitems} % Description(s) bullet points
        \item {Research Geneticist}
        \item {JeanLuc.Jannink@ars.usda.gov}
        \item {607-255-5266}
      \end{cvitems}
    }

%---------------------------------------------------------
    
\end{cventries}
%

%-------------------------------------------------------------------------------
\end{document}
